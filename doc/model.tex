\documentclass[10pt]{article}
\usepackage[utf8]{inputenc}
\usepackage[english]{babel}
\usepackage{graphicx}
\usepackage{amsmath}
\usepackage{amssymb}
\usepackage{hyperref}

\newcommand{\dd}{\mathrm{d}}
\renewcommand{\vec}[1]{\mathbf{#1}}
\newcommand{\ve}{\vec{e}}
\newcommand{\vi}{\vec{i}}
\newcommand{\vk}{\vec{k}}
\newcommand{\vK}{\vec{K}}
\newcommand{\vn}{\vec{n}}
\newcommand{\vx}{\vec{x}}
\newcommand{\fFD}{f_{\mathrm{FD}}}
\newcommand{\fL}{f_{\mathrm{L}}}
\newcommand{\fG}{f_{\mathrm{G}}}
\newcommand{\ii}{\mathrm{i}}

\begin{document}

\title{ARPES signal calculation in a slab with a depletion/accumulation
surface layer (quasi-1D~model)}
\author{https://github.com/eugnsp/surface\_arpes}
\maketitle

This document conains a succinct description of equations that are numerically solved
in the code. No attempt has been made to present mathematically and/or physically
rigorous derivations. Please see references for details.

\section{Quasi-classical problem}

\subsection{Poisson equation}

To find the potential profile~$\phi(z)$, we solve the 1D Poisson equation:
\begin{equation}
	\label{eq:poisson}
	- \frac{\dd}{\dd z} \left[ \epsilon \frac{\dd\phi(z)}{\dd z} \right]
	= 4\pi [N_D - n(z)],
\end{equation}
where $N_D$ is the (spatially uniform) density of ionized donors, and $n(z)$ is the
electron density. It is given by
\begin{equation}
	\label{eq:c_density}
	n(z) = N_c^\mathrm{3D} \fFD^{1/2} \left[ -\frac{E_c(z) - F}{T} \right]
	     = N_c^\mathrm{3D} \fFD^{1/2} \left[ -\frac{E_{c0} - \phi(z) - F}{T} \right]
\end{equation}
where $N_c$ is the effective 3D density of states, $\fFD^{\alpha}$ is the Fermi--Dirac
integral of order~$\alpha$,
\begin{equation}
	\fFD^{\alpha}(x) =
	\frac{1}{\Gamma(\alpha + 1)} \int_0^\infty dt\, \frac{t^\alpha}{1 + e^{t - x}},
\end{equation}
$F$ is the Fermi level, $E_{c0}$ if the conduction band minimum,
and $T$ is the lattice temperature.

The effective 3D density of states is given by
\begin{equation}
	N_c^\mathrm{3D} = 2 \left[ \frac{m T}{2\pi} \right]^{3/2},
\end{equation}
where $m$ is the electron effective mass.

The zero level of energy is chosen such that $E_{c0} = 0$. Then the arbitrary additive
constant in the potential~$\phi(x)$ is chosen such that $\phi(x) = 0$ corresponds to
charge neutrality:
\begin{equation}
	N_D = N_c^\mathrm{3D} \fFD^{1/2} \left[ \frac{F}{T} \right],
	\quad\text{or}\quad
	F = T \{\fFD^{1/2}\}^{-1} \left[ \frac{N_D}{N_c^\mathrm{3D}} \right].
\end{equation}

Finally:
\begin{equation}
	n(\phi) = N_c^\mathrm{3D} \fFD^{1/2} \left[ \frac{\phi + F}{T} \right], \quad
	\frac{\dd n}{\dd \phi} =
		\frac{N_c^\mathrm{3D}}{T} \fFD^{-1/2} \left[ \frac{\phi + F}{T} \right].
\end{equation}

The Poisson equation~\eqref{eq:poisson} is a non-linear equation. To solve it, we employ
the Newton's method. Linearizing for $\phi(z) = \phi_0(z) + \delta\phi(z)$, we get:
\begin{equation}
	\label{eq:poisson_newton}
	- \frac{\dd}{\dd z} \left[ \epsilon \frac{\dd \delta\phi(z)}{\dd z} \right]
	+ 4\pi \frac{\dd n_0}{\dd \phi_0} \delta\phi(z)
	= \frac{\dd}{\dd z} \left[ \epsilon \frac{\dd \phi_0(z)}{\dd z} \right]
	+ 4\pi [N_D - n_0(z)],
\end{equation}
where $n_0(z) = n(\phi_0(z))$.

\textbf{Boundary conditions.} The boundary conditions for eq.~\eqref{eq:poisson} are:
\begin{equation}
	\phi(0) = \phi_b, \quad \frac{\dd \phi(L)}{\dd z} = E(L) = 0,
\end{equation}
where $L$ is the system length, chosen such that $L \gg$ screening length. Corresponding
boundary conditions for eq.~\eqref{eq:poisson_newton} are:
\begin{equation}
	\delta\phi(0) = 0, \quad \frac{\dd \delta\phi(L)}{\dd z} = 0.
\end{equation}

The initial guess for~$\phi_0(z)$ is obtained from local charge neutrality inside the
system. Hence, we start Newton's iterations from $\phi_0(0) = \phi_b$, $\phi_0(z > 0) = 0$.

\textbf{Finite elements discretization.} Galerkin's method is used for numerical solution
of eq.~\eqref{eq:poisson_newton}. We multiply it by the test function~$\psi_i(z)$ and
integrate by parts taking into account the boundary conditions:
\begin{equation}
	\int \frac{\dd \psi_i}{\dd z} \frac{\dd \delta\phi}{\dd z}
	+ 4\pi \int \frac{\dd n_0}{\dd \phi_0} \psi_i \delta\phi
	= - \int \epsilon \frac{\dd \psi_i}{\dd z} \frac{\dd \phi_0}{\dd z}
	+ 4\pi \int (N_D - n_0) \psi_i.
\end{equation}

Expanding $\delta\phi(z)$ over $\psi_i(z)$, $\delta\phi(x) = \sum_j \delta\phi_j \psi_j(z)$,
we obtain the matrix equation:
\begin{equation}
	\sum_j \left( S_{ij} - M_{ij} \right) \delta\phi_j = r_i
\end{equation}
with
\begin{equation}
\begin{split}
	S_{ij} &= \int \epsilon \frac{\dd \psi_i}{\dd z} \frac{\dd \psi_j}{\dd z}, \quad
	M_{ij} = -4\pi \int \frac{\dd n_0}{\dd \phi_0} \psi_i \psi_j, \\
	r_i &= - \sum_j S_{ij} \phi_{0j} + 4\pi \int (N_D - n_0) \psi_i.
\end{split}
\end{equation}

Integrals are calculated using Gauss quadratures $\{z_k, w_k\}$. In particular:
\begin{equation}
	\int n_0 \psi_i = \sum_k w_k n[\phi_0(z_k)] \psi_i(z_k)
	= \sum_k w_k n\Big[ \sum_l \phi_{0l} \psi_l(z_k) \Big] \psi_i(z_k),
\end{equation}
and
\begin{equation}
	\int \frac{\dd n_0}{\dd \phi_0} \psi_i \psi_j
	= \sum_k w_k \frac{\dd n}{\dd \phi}
		\Big[ \sum_l \phi_{0l} \psi_l(z_k) \Big] \psi_i(z_k) \psi_j(z_k).
\end{equation}

\section{Quantum problem}

To account for quantum effects, we solve the Schr\"odinger and the Poisson
equations self-consistently. First, we solve the Schr\"odinger equation
\begin{equation}
	\label{eq:schrodinger}
	\frac{1}{2m} \frac{\dd^2}{\dd z^2} \psi(z) + V(z) \psi(z) = E \psi(z)
\end{equation}
with $V(z) = -\phi(z)$, and then use the first-order perturbation theory to turn a
linear Poisson equation into a non-linear one with the electron density given by
\begin{equation}
	n(\phi, z) = N_c^\mathrm{2D} \sum_n \lvert\psi_n^{(0)}(z)\rvert^2 \fFD^0
				 \left[ \frac{\phi(z) - \phi^{(0)}(z) + F - E_n^{(0)}}{T} \right],
\end{equation}
where the effective 2D density of states is given by
\begin{equation}
	N_c^\mathrm{2D} = \frac{m T}{\pi}.
\end{equation}
and summation is performed over all eigenstates~$\{ \psi_n^{(0)}(z), E_n^{(0)} \}$
that were obtained by solving \eqref{eq:schrodinger} with the potential
$V(z) = -\phi_0^{(0)}(z)$. The derivative of $n(\phi, z)$ is
\begin{equation}
	\frac{\dd n}{\dd \phi} =
		\frac{N_c^\mathrm{2D}}{T} \sum_n \lvert\psi_n^{(0)}(z)\rvert^2 \fFD^{-1}
		\left[ \frac{\phi(z) - \phi^{(0)}(z) + F - E_n^{(0)}}{T} \right].
\end{equation}

The initial approximation for $\phi(z)$ is obtained from the solution of the
quasi-classical problem \eqref{eq:poisson}--\eqref{eq:c_density}.

The Fermi--Dirac integrals $\fFD^0(x)$ and $\fFD^{-1}(x)$ have closed forms:
\begin{equation}
	\fFD^0(x) = \ln[ 1 + e^x ], \quad
	\fFD^{-1}(x) = \frac{1}{1 + e^{-x}} = f_{\mathrm{FD}}(-x),
\end{equation}
where $f_{\mathrm{FD}}(x)$ is the Fermi--Dirac distribution.

\textbf{Boundary conditions.} The potential barriers at the system
boundaries are assumed to be infinitely high, so we put
\begin{equation}
	\psi(z = 0) = \psi(z = L) = 0.
\end{equation}

\textbf{Finite elements discretization.} Here we also use the Galerkin's
method. Eq.~\eqref{eq:schrodinger} reduces to a generalized eigenvalue problem:
\begin{equation}
	\sum_j \left( T_{ij} + V_{ij} \right) \psi_j = E_n B_{ij} \psi_j
\end{equation}
with
\begin{equation}
	T_{ij} = \frac{1}{2m }\int \frac{\dd \psi_i}{\dd z} \frac{\dd \psi_j}{\dd z},
	\quad
	V_{ij} = \int V \psi_i \psi_j, \quad B_{ij} = \int \psi_i \psi_j,
\end{equation}
where $B$ is positive-definite.

\section{ARPES signal calculation}

The intensity of the ARPES signal at $k_y = 0$ is given by:
\begin{multline}
	I_0(E, k_x, k_z) \sim f_{\mathrm{FD}}\left[ \frac{E - F}{T} \right] \times\\
		\times \sum_n f_{\mathrm{D}}\left[ E - E_n - \frac{k_x^2}{2m} \right]
		\left\vert \int \dd z \, e^{-\ii k_z z - z / \lambda}
		\psi_n(z) \right\vert^2,
\end{multline}
where the Lorentz distribution
\begin{equation}
	f_{\mathrm{D}}(E) \sim \frac{1}{1 + (E / \delta_D E)^2}
\end{equation}
accounts for disorder.

If $\phi(z) = 0$, in the limit $L \to \infty$:
\begin{equation}
	I_0(E, k_x, k_z) \sim f_{\mathrm{FD}}\left[ \frac{E - F}{T} \right]
		\int \dd k_z' \, f_{\mathrm{D}}\left[ E - \frac{k_x^2 + k_z'^2}{2m} \right]
		\frac{1}{1 + [\lambda (k_z - k_z')]^2}.
\end{equation}

In the limit $\lambda \to \infty$ we recover a broadened parabolic spectrum:
\begin{equation}
	I_0(E, k_x, k_z) \sim f_{\mathrm{FD}}\left[ \frac{E - F}{T} \right]
		f_{\mathrm{D}}\left[ E - \frac{k_x^2 + k_z^2}{2m} \right].
\end{equation}

At $k_z = 0$ we have:
\begin{equation}
	I_0(E, k_x) \sim f_{\mathrm{FD}}\left[ \frac{E - F}{T} \right] \sum_n
		f_{\mathrm{D}}\left[ E - E_n - \frac{k_x^2}{2m} \right]
		\left[ \int \dd z \, e^{- z / \lambda}
		\psi_n(z) \right]^2.
\end{equation}

To account for apparatus broadenings, we futher convolve $I_0(E, k_x)$ with
two normal distributions
\begin{equation}
	f_A(E) \sim \exp\left[ -\frac{E^2}{2(\delta_A E)^2} \right], \quad
	f_A(k) \sim \exp\left[ -\frac{k^2}{2(\delta_A k)^2} \right]
\end{equation}
to obtain the final result:
\begin{multline}
	I(E, k_x) \sim I_0(E, k_x) \star (f_A(E) \otimes f_A(k)) = \\
	= \int \dd k_x' \dd E' \, f_A(E') f_A(k_x') I_0(E - E', k_x - k_x').
\end{multline}

% The single-particle density of a filled state at
% fixed depth~$z$ and fixed momentum $k_y = 0$ is given by the product
% \begin{eqnarray}
% 	I(E, k_x, z) = f_{\mathrm{L}}\left[ E - \frac{k_x^2}{2m} + \phi(z) \right]
% 				   f_{\mathrm{FD}}\left( \frac{E - F}{T} \right),
% \end{eqnarray}
% of the Lorentzian distribution $f_{\mathrm{L}}(E)$ with half width at half
% maximum $\delta_E$,
% \begin{eqnarray}
% 	f_{\mathrm{L}}(E) = \frac{1}{\pi \delta_E}
% 		\frac{1}{1 + \left( \dfrac{E}{\delta_E} \right)^2},
% \end{eqnarray}
% and the Fermi--Dirac distribution.

% \textbf{Quantum.} The expression is:
% \begin{eqnarray}
% 	I(E, k_x, z) = \sum_n f_{\mathrm{L}}\left[ E - \frac{k_x^2}{2m} - E_n \right]
% 				   f_{\mathrm{FD}}\left( \frac{E - F}{T} \right)
% 				   \lvert \psi_n(z) \rvert^2.
% \end{eqnarray}

% The total ARPES signal is obtained by integration over~$z$. Assuming that the
% contribution of states at the depth~$z$ is proportional to $\exp(-z / \lambda)$,
% we obtain:
% \begin{equation}
% 	I(E, k_x) \sim \int_0^\infty \dd z \exp\left( -\frac{z}{\lambda} \right)
% 				f_{\mathrm{L}}\left[ E - \frac{k_x^2}{2m} + \phi(z) \right]
% 				f_{\mathrm{FD}}\left( \frac{E - F}{T} \right),
% \end{equation}
% or
% \begin{equation}
% 	I(E, k_x) \sim \int_0^\infty \dd z \exp\left( -\frac{z}{\lambda} \right)
% 				\sum_n f_{\mathrm{L}}\left[ E - \frac{k_x^2}{2m} + E_n \right]
% 				f_{\mathrm{FD}}\left( \frac{E - F}{T} \right)
% 				\lvert \psi_n(z) \rvert^2.
% \end{equation}

\end{document}
